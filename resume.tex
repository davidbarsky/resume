%!TEX TS-program = xelatex
%!TEX encoding = UTF-8 Unicode

\documentclass[10pt, a4paper]{article}

% LAYOUT
%--------------------------------
% Margins
\usepackage{geometry}
\geometry{letterpaper, left=40mm, right=40mm, top=16mm, bottom=16mm}

% Do not indent paragraphs
\setlength\parindent{0in}

% TYPOGRAPHY
%--------------------------------
\usepackage{fontspec}
\usepackage{xunicode}
\usepackage{xltxtra}
% converts LaTeX specials (quotes, dashes etc.) to Unicode
\defaultfontfeatures{Mapping=tex-text}
\setromanfont [Ligatures={Common}]{Inter UI}
% Cool ampersand
\newcommand{\amper}{{\fontspec[Scale=.95]{Inter UI}\selectfont\itshape\&}}
\setmainfont[
    BoldFont = * Semi Bold,
]{Inter UI}
\raggedright

% MARGIN NOTES
%--------------------------------
\usepackage{marginnote}
\newcommand{\note}[1]{\marginnote{\scriptsize #1}}
\renewcommand*{\raggedleftmarginnote}{}
\setlength{\marginparsep}{14pt}
\setlength{\marginparwidth}{70pt}
\reversemarginpar
\pagenumbering{gobble}

% HEADINGS
%--------------------------------
\usepackage{sectsty}
\usepackage[normalem]{ulem}

\sectionfont{\rmfamily\mdseries}
\subsectionfont{\rmfamily\mdseries\scshape\normalsize}
\subsubsectionfont{\rmfamily\bfseries\upshape\normalsize}

% PDF SETUP
%--------------------------------
\usepackage{hyperref}
\hypersetup
{
  pdfauthor={David Barsky},
  pdfsubject={David Barsky's Resume},
  pdftitle={David Barsky's Resume},
  colorlinks, breaklinks, xetex, bookmarks,
  filecolor=black,
  urlcolor=[rgb]{0.117,0.682,0.858},
  linkcolor=[rgb]{0.117,0.682,0.858},
  linkcolor=[rgb]{0.117,0.682,0.858},
  citecolor=[rgb]{0.117,0.682,0.858}
}

% LIST ENVIRONMENT
%--------------------------------
\usepackage{paralist}
\setdefaultleftmargin{.5cm}{2cm}{}{}{}{}
\renewenvironment{itemize}[1]{\begin{compactitem}#1}{\end{compactitem}}

% DOCUMENT
%--------------------------------
\begin{document}

{\LARGE\textbf{David Barsky}}\\[.2cm]

\href{mailto:me@davidbarsky.com}{me@davidbarsky.com}\\
\href{http://davidbarsky.com}{davidbarsky.com}\\
\href{http://github.com/davidbarsky}{github.com/davidbarsky}\\
\href{http://linkedin.com/in/davidbarsky}{linkedin.com/in/davidbarsky}

\section*{\textbf{Work Experience}}

\note{Fall 2017–Present}
\textbf{Amazon, Alexa AI} \\
\emph{Software Engineer}
\begin{itemize}
	\item Integrated test data generation system into unified first/third party natural language model building/testing platform, removing distinction between and first and third-party skill development.
	\item Sped up test data validation tooling used by customers 22x, reducing runtime from 45 minutes to 2 minutes.
	\item Maintaining Rust-specific integration to Amazon-internal build/deployment systems.
    \item Launched the AWS Lambda Runtime for Rust.
	\item Designed and implemented a test data generation service used across Alexa. This new system is horizontally scalable and is 7x faster over the prior system.
    \item Implemented Alexaʼs natural language model to be hardware-capability aware. This significantly reduced engineering effort needed to support new devices and cut the public Alexa serviceʼs memory usage by 30\%. 
\end{itemize}

\note{Summer 2016}
\textbf{Amazon, AWS Payments} \\
\emph{Software Engineer (Intern)}
\begin{itemize}
	\item Worked in Payments organization in Amazon Web Services.
    \item Architected and developed a distributed, fault-tolerant service for financial data auditing that simplified payments infrastructure and reduced on-call burden.
\end{itemize}

\section*{\textbf{Education}}
\note{Fall 2013—2017}\textbf{Brandeis University}\\
Computer Science, B.A. with Honors. \\
Completed additional coursework in History and Politics

\section*{\textbf{Selected Projects}}


\textbf{Rust Runtime for AWS Lambda} (\href{https://github.com/awslabs/aws-lambda-rust-runtime}{github.com/awslabs/aws-lambda-rust-runtime})
\begin{itemize}
    \item Launched and maintaining the Rust Runtime for AWS Lambda.
\end{itemize}

\textbf{Tokio} (\href{https://github.com/tokio-rs/tokio}{github.com/tokio-rs/tokio})
\begin{itemize}
    \item Co-maintainer of Tokio, Rust's asynchronous runtime.
    \item Helped launch Tracing, a unified, high-performance instrumentation system for Rust that emits logs, metrics, and traces.
\end{itemize}

\textbf{Honor's Thesis} (\href{https://github.com/davidbarsky/sirens}{github.com/davidbarsky/sirens})
\begin{itemize}
    \item Benchmarked job scheduling algorithms across various simulated workloads.
\end{itemize}

\section*{\textbf{Skills}}
\textbf{Languages:} Rust, Go, Python, Swift, Java \\
\textbf{Tooling:} AWS (DynamoDB, ECS, Lambda), Unix-like systems, Docker, Git, gRPC

\end{document}
