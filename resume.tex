%!TEX TS-program = xelatex
%!TEX encoding = UTF-8 Unicode

\documentclass[10pt, a4paper]{article}

% LAYOUT
%--------------------------------
% Margins
\usepackage{geometry}
\geometry{letterpaper, left=40mm, right=36mm, top=10mm, bottom=10mm}

% Do not indent paragraphs
\setlength\parindent{0in}

% TYPOGRAPHY
%--------------------------------
\usepackage{fontspec}
\usepackage{xunicode}
\usepackage{xltxtra}
% converts LaTeX specials (quotes, dashes etc.) to Unicode
\defaultfontfeatures{Mapping=tex-text}
\setromanfont [Ligatures={Common}]{Source Serif Pro}
% Cool ampersand
\newcommand{\amper}{{\fontspec[Scale=.80]{Source Serif Pro}\selectfont\itshape\&}}
\setmainfont{Source Serif Pro}
\raggedright

% MARGIN NOTES
%--------------------------------
\usepackage{marginnote}
\newcommand{\note}[1]{\marginnote{\scriptsize #1}}
\renewcommand*{\raggedleftmarginnote}{}
\setlength{\marginparsep}{10pt}
\setlength{\marginparwidth}{70pt}
\reversemarginpar
\pagenumbering{gobble}

% HEADINGS
%--------------------------------
\usepackage{sectsty}
\usepackage[normalem]{ulem}

\sectionfont{\rmfamily\mdseries}
\subsectionfont{\rmfamily\mdseries\scshape\normalsize}
\subsubsectionfont{\rmfamily\bfseries\upshape\normalsize}

% PDF SETUP
%--------------------------------
\usepackage{hyperref}
\hypersetup
{
  pdfauthor={David Barsky},
  pdfsubject={David Barsky's Resume},
  pdftitle={David Barsky's Resume},
  colorlinks, breaklinks, xetex, bookmarks,
  filecolor=black,
  urlcolor=[rgb]{0.117,0.682,0.858},
  linkcolor=[rgb]{0.117,0.682,0.858},
  linkcolor=[rgb]{0.117,0.682,0.858},
  citecolor=[rgb]{0.117,0.682,0.858}
}

% LIST ENVIRONMENT
%--------------------------------
\usepackage{paralist}
\setdefaultleftmargin{.5cm}{1cm}{}{}{}{}
\renewenvironment{itemize}[1]{\begin{compactitem}#1}{\end{compactitem}}

% DOCUMENT
%--------------------------------
\begin{document}

{\LARGE\textbf{David Barsky}}\\[.1cm]

\href{mailto:me@davidbarsky.com}{me@davidbarsky.com}\\
\href{http://davidbarsky.com}{davidbarsky.com}\\
\href{http://github.com/davidbarsky}{github.com/davidbarsky}\\
\href{http://linkedin.com/in/davidbarsky}{linkedin.com/in/davidbarsky}

\section*{\textbf{Work Experience}}

\note{Fall 2017–Present}
\textbf{Amazon, Alexa AI} \\
\emph{Software Engineer}
\begin{itemize}
	\item Integrated internal test data generation system into a unified, self-service natural language model building and testing platform.
	\item Enabled multiple Alexa verticals (e.g., Music, Shopping) to release features independently and concurrently, doubling daily release capacity.
	\item Drove AWS to sponsor the Rust project's AWS bill for the next three years.
	\item Reduced execution time of NLU model validation tooling from 45 minutes to 2 minutes.
	\item Reduced execution time of a test-data generation system from 12 hours to 15 minutes.
	\item Mentored three interns, who have all returned to Amazon as full-time SDEs.
	\item Implemented Alexaʼs natural language model to be hardware-capability aware. This significantly reduced engineering effort needed to support new devices and cut the public Alexa serviceʼs memory usage by 30\%. 
\end{itemize}

\note{Summer 2016}
\textbf{Amazon, AWS Payments} \\
\emph{Software Engineer (Intern)}
\begin{itemize}
	\item Worked in Payments organization in Amazon Web Services.
    \item Architected and developed a distributed, fault-tolerant service for financial data auditing that simplified payments infrastructure and reduced on-call burden.
\end{itemize}

\section*{\textbf{Education}}
\note{Fall 2013—2017}\textbf{Brandeis University}\\
Computer Science, B.A. with Honors. \\
Completed additional coursework in History and Politics

\section*{\textbf{Selected Projects}}

\textbf{Tracing} (\href{https://github.com/tokio-rs/tracing}{github.com/tokio-rs/tracing})
\begin{itemize}
    \item Helped launch Tracing, a unified, high-performance instrumentation system for Rust that emits logs, metrics, and traces.
    \item Designed and wrote a lock-free data store for storing high-cardinality instrumentation data. This store is now the basis of several third-party instrumentation exporters, such as OpenTelemetry, Prometheus, and Amazon's internal metrics format.
    \item Successfully introduced Tracing into several key AWS services, whose total request volume exceeds two million requests per second.
\end{itemize}

\textbf{Tower} (\href{https://github.com/tokio-rs/tower}{github.com/tower-rs/tower})
\begin{itemize}
    \item Contributed documentation and more ergonomic error handling to Tower, a library for unified client/server middleware.
    \item Built an HTTP congestion control library that expands/shrinks depending on downstream service health.
\end{itemize}

\textbf{Rust Runtime for AWS Lambda} (\href{https://github.com/awslabs/aws-lambda-rust-runtime}{github.com/awslabs/aws-lambda-rust-runtime})
\begin{itemize}
    \item Launched the Rust Runtime for AWS Lambda.
    \item Converted project to be asynchronous async/await, enabling concurrent executions and easier in-memory testing of Lambda functions.
\end{itemize}
\section*{\textbf{Skills}}
\textbf{Languages:} Rust, Java, Go, Kotlin, Python. \\
\textbf{Tooling:} AWS (DynamoDB, ECS, Lambda), Linux, Docker, Git, gRPC.

\end{document}
