%!TEX TS-program = xelatex
%!TEX encoding = UTF-8 Unicode

\documentclass[11pt, a4paper]{article}

% LAYOUT
%--------------------------------
% Margins
\usepackage{geometry}
\geometry{letterpaper, left=40mm, right=40mm, top=16mm, bottom=16mm}

% Do not indent paragraphs
\setlength\parindent{0in}

% TYPOGRAPHY
%--------------------------------
\usepackage{fontspec}
\usepackage{xunicode}
\usepackage{xltxtra}
% converts LaTeX specials (quotes, dashes etc.) to Unicode
\defaultfontfeatures{Mapping=tex-text}
\setromanfont [Ligatures={Common}]{Minion Pro}
% Cool ampersand
\newcommand{\amper}{{\fontspec[Scale=.95]{Minion Pro}\selectfont\itshape\&}}
\raggedright

% MARGIN NOTES
%--------------------------------
\usepackage{marginnote}
\newcommand{\note}[1]{\marginnote{\scriptsize #1}}
\renewcommand*{\raggedleftmarginnote}{}
\setlength{\marginparsep}{14pt}
\setlength{\marginparwidth}{66pt}
\reversemarginpar
\pagenumbering{gobble}

% HEADINGS
%--------------------------------
\usepackage{sectsty}
\usepackage[normalem]{ulem}

\sectionfont{\rmfamily\mdseries}
\subsectionfont{\rmfamily\mdseries\scshape\normalsize}
\subsubsectionfont{\rmfamily\bfseries\upshape\normalsize}

% PDF SETUP
%--------------------------------
\usepackage{hyperref}
\hypersetup
{
  pdfauthor={David Barsky},
  pdfsubject={David Barsky's Resume},
  pdftitle={David Barsky's Resume},
  colorlinks, breaklinks, xetex, bookmarks,
  filecolor=black,
  urlcolor=[rgb]{0.117,0.682,0.858},
  linkcolor=[rgb]{0.117,0.682,0.858},
  linkcolor=[rgb]{0.117,0.682,0.858},
  citecolor=[rgb]{0.117,0.682,0.858}
}

% LIST ENVIRONMENT
%--------------------------------
\usepackage{paralist}
\setdefaultleftmargin{.5cm}{2cm}{}{}{}{}
\renewenvironment{itemize}[1]{\begin{compactitem}#1}{\end{compactitem}}

% DOCUMENT
%--------------------------------
\begin{document}

{\LARGE David Barsky}\\[.2cm]

\href{mailto:me@davidbarsky.com}{me@davidbarsky.com}\\
\href{http://davidbarsky.com}{davidbarsky.com}\\
\href{http://github.com/davidbarsky}{github.com/davidbarsky}\\
\href{http://linkedin.com/in/davidbarsky}{linkedin.com/in/davidbarsky}

\section*{Education}
\note{Fall 2013—2017}\textbf{Brandeis University}\\
Computer Science, B.A. with Honors. \\
Completed additional coursework in History and Politics

\section*{Work Experience}

\note{Fall 2017–Present}
\textbf{Amazon} \\
\emph{Software Engineer}
\begin{itemize}
    \item Worked on internal machine learning tooling for Alexa NLU.
    \item Implemented GDPR-compliant retention policies on two petabytes of user-derived test and training data.
    \item Moved dependencies across Alexa NLU onto maintained artifact repository, allowing engineers to move off deprecated packages.
    \item Migrated tooling and services to rely on a secure compute platform that reduced model build and test execution time by 50\%.
\end{itemize}

\note{Summer 2016}
\textbf{Amazon Web Services} \\
\emph{Software Engineer (Intern)}
\begin{itemize}
    \item Architected and developed a distributed, fault-tolerant service for financial data auditing that simplified payments infrastructure and reduced on-call burden.
\end{itemize}

\note{Summer 2015}
\textbf{Localytics} \\
\emph{Software Engineer (Intern)}
\begin{itemize}
    \item Developed an internal iOS alpha of the core dashboard product using functional reactive programming techniques.
\end{itemize}

\section*{Selected Projects}

% \note{Winter 2017}

% \textbf{Boston Rust Meetup at Amazon}
% \begin{itemize}
%     \item Maneuvered corporate bureaucracy to host a Rust meetup at Amazon.
%     \item Spoke about integrating Rust into proprietary build systems.
% \end{itemize}

\note{Spring 2018}
\textbf{rustfmt} (\href{https://github.com/rust-lang-nursery/rustfmt}{github.com/rust-lang-nursery/rustfmt})
\begin{itemize}
    \item Implemented functionality for checking correctness of formatting in continuous integration-like enviroments.
\end{itemize}

\note{Winter 2018}
\textbf{fs-sync} (\href{https://github.com/davidbarsky/fs-sync}{github.com/davidbarsky/fs-sync})
\begin{itemize}
    \item Designed and implemented an intelligent rsync clone in Rust.
\end{itemize}

\note{Fall 2016–Spring 2017}
\textbf{Honor's Thesis} (\href{https://github.com/davidbarsky/sirens}{github.com/davidbarsky/sirens})
\begin{itemize}
    \item Studying efficient task scheduling using probabilistic techniques. Made extensive use of Scala's functional features to ensure correctness.
\end{itemize}

\section*{Skills and Interests}
\textbf{Interests:} Distributed systems, functional programming, service observability \\
\textbf{Languages:} Rust, Python, Go, Java, Scala, SQL \\
\textbf{Tooling:} Amazon Web Services, Unix systems, Docker, Git, RESTful APIs, \LaTeX

% \section*{Academics}

% \textbf{Classwork:} Database Management Systems, Operating Systems, Structure and Interpretation of Computer Programs, Data Structures and Algorithms \\
% \textbf{Activities:} Parliamentary Debate, Rock Climbing, Rowing

\end{document}
